\documentclass[12pt]{article}
\usepackage[a4paper, text={6.5in,9in}]{geometry}
% \usepackage[utf8]{inputenc}
% \usepackage{graphicx}
% \graphicspath{ {immagini} }

% \usepackage{titling}

\usepackage{hyperref}
\hypersetup{
    colorlinks,
     citecolor=black,
     filecolor=black,
     linkcolor=black,
     urlcolor=black
 }

% \usepackage{fancyhdr}
% \pagestyle{fancy}

\usepackage{amsmath}
\usepackage{amssymb}
\usepackage{mathtools}
\usepackage{dsfont}
\usepackage{cases}
% \newcommand*{\Z}{\mathds{Z}}

\usepackage{minted, xcolor}
%\usemintedstyle{monokai}
\definecolor{bg}{HTML}{F0F0F0}
% \usepackage[defaultmono]{droidsansmono}
% \usepackage[T1]{fontenc}

% \pretitle{%
%   \begin{center}
%   \LARGE
%   \includegraphics[width=6cm]{logo-unipd}\\[\bigskipamount]
% }
% \posttitle{\end{center}}

\title{\textbf{Università di Padova \\ Formal Methods for Cyber-Physical Systems project report}}
\author{Alberto Lazari - 1216747\\}
\date{Giugno 2022 - A.A. 2021-2022}

\renewcommand*\contentsname{Indice}

\begin{document}
    \maketitle
    \pagebreak

    \tableofcontents
    \pagebreak

    \section{Notazione}
    \begin{description}
        \item[$Post$]: funzione che rappresenta la regione di stati raggiungibili da una regione data applicando un solo passo di transizione.
    \end{description}

    \section{Reachability}
    \subsection{Idea di base}
    L'obiettivo di questo algoritmo è, dati una regione di stati iniziali $Init$, una funzione di transizione $Trans$ e un invariante $Inv$, decidere se $Inv$ è verificato in tutti gli stasti raggiungibili a partire da $Init$ applicando $Trans$.

    L'idea per fare ciò è trovare una rappresentazione simbolica di tutti e soli gli stati raggiungibili dal sistema.
    Così facendo risulta semplice verificare se in alcuni di questi stati l'invariante non è verificato.
    È possibile utilizzare un'ottimizzazione per evitare di calcolare tutti gli stati nel caso in cui nel processo vengano trovati degli stati che non soddisfano l'invariante.

    Per fare ciò, utilizziamo una variabile $Reach$ che rappresenta gli stati raggiungibili dal sistema e la aggiorniamo iterativamente.
    L'idea di base è che all'iterazione $t$, $Reach$ rappresenta tutti gli stati raggiungibili a partire da $Init$ e applicando $Trans$ al più $t$ volte.
    L'algoritmo si interrompe quando si verifica una delle due seguenti condizioni:
    \begin{enumerate}
        \item $Post(Reach) \subseteq Reach $: abbiamo trovato tutti i possibili stati raggiungibili;
        \item $Reach \cap NotInv \neq \emptyset$: abbiamo trovato alcuni stati raggiungibili in cui non vale l'invariante.
    \end{enumerate}

    %TODO
    \textbf{TODO Disegnino}

    \subsection{Implementazione}
    Di seguito viene riportato lo pseudocodice dell'algoritmo già descritto:

    %TOEND
    \begin{minted}[bgcolor=bg, breaklines, fontsize=\small, mathescape=true, escapeinside=||, linenos]{python}
function IsInvariantRespected(Init, Trans, Inv)
    NotInv |$\leftarrow \mathbb U\ \setminus$| Inv
    Reach |$\leftarrow$| Init
    New |$\leftarrow$| Init
    while New |$\neq \varnothing$| do
        if New |$\cap$| NotInv |$\neq \varnothing$| then
            return False
        end if
        New |$\leftarrow$| Post(New, Trans) |$\setminus$| Reach
        Reach |$\leftarrow$| Reach |$\cup$| New
    end while
    return True
end function
    \end{minted}

    Per l'Implementazione effettiva in python è sufficiente tradurre le seguenti istruzioni:
    \begin{itemize}
        \item $A \leftarrow B$ diventa: \mintinline{python}{A = B}
        \item $\mathbb U\ \setminus A$ diventa: \mintinline{python}{~A}
        \item $A \neq \varnothing$ diventa: \mintinline{python}{not A.is_false()}
        \item $A \cap B \neq \varnothing$ diventa: \mintinline{python}{A.intersected(B)}
        \item $Post(A, Trans)$ diventa: \mintinline{python}{model.post(A)}
        \item $A \setminus B$ diventa: \mintinline{python}{A - B}
        \item $A \cup B$ diventa: \mintinline{python}{A + B}
    \end{itemize}

    %TODO spiegare operatori

    %TODO maybe pensare se mettere il codice easy python
    
    \subsection{Dimostrazione di correttezza}
    Definiamo:
    \begin{itemize}
        \item $\mathbb U$ l'insieme di tutti i possibili stati del modello;
        \item $Inv \subseteq \mathbb U$ insieme degli stati che rispettano l'invariante;
        \item $NotInv \vcentcolon= \mathbb U \setminus Inv \subseteq \mathbb U$;
        \item $Reach^k$ è l'insieme degli stati raggiungibili in al più $k$ step, definito in modo tale che:
        $$
            \begin{cases}
                Reach^0 = Init \\
                Reach^{k + 1} = Reach^k \cup Post(Reach^k)
            \end{cases}
        $$
        \item $New^k$ è l'insieme degli stati raggiungibili in esattamente $k$ step, definito in modo tale che:
        $$
            \begin{cases}
                New^0 = Init \\
                New^{k + 1} = Reach^{k+1} \setminus Reach^k
            \end{cases}
        $$
    \end{itemize}
    In questo modo, $Reach^k$ e $New^k$ corrispondono ai valori assunti dalle variabili \texttt{Reach} e \texttt{New} alla $k$-esima iterazione.

    Vogliamo quindi dimostrare che:
    \begin{itemize}
        % \item le definizioni date di $New^k$ e $Reach^k$ sono sound;
        \item l'algoritmo termina sempre;
        \item $Reach^k \subseteq Inv\ \forall k$ se l'algoritmo ritorna true;
        \item $\exists k \mbox{ s.t. } Reach^k \cap NotInv \neq \varnothing$ se l'algoritmo ritorna false.
    \end{itemize}

    % \paragraph{Soundness of definitions}
    % \newcommand{\New}{\mathtt{New}}
    % \newcommand{\Reach}{\mathtt{Reach}}
    % Dimostriamo per induzione su $k$ che la definizione di $New$ e $Reach$ è corretta.
    % In particolare, dimostriamo che $New^k$ e $Reach^k$ corrispondono al valore delle variabili $\New^k$ e $\Reach^k$ al $k$-esimo ciclo dell'algoritmo.
    % Ovvero, vogliamo dimostrare che
    % \begin{equation}
    %     New^k = \New^k \wedge Reach^k = \Reach^k\ \forall k
    % \end{equation}

    % \subparagraph*{Caso base:}
    % $New^0 = Reach^0 = Init$ è corretto poiché corrisponde all'inizializzazione delle variabili nell'algoritmo.

    % \subparagraph*{Caso induttivo ($k+1$):}
    % Seguendo il flusso dell'algoritmo, assumendo che non sia già terminato, otteniamo:
    % \begin{equation}\label{th:sound:def_var}
    %     \begin{cases}
    %         \New^{k+1} = Post(\New^k) \setminus \Reach^k \\
    %         \Reach^{k+1} = \Reach^k \cup \New^k
    %     \end{cases}
    % \end{equation}

    % Inoltre, per definizione di $New$ e $Reach$ sappiamo che:
    % \begin{equation}\label{th:sound:def_ind}
    %     \begin{cases}
    %         Reach^{k+1} = Reach^k \cup Post(Reach^k) \\
    %         New^{k+1} = Reach^{k+1} \setminus Reach^k 
    %     \end{cases}
    % \end{equation}

    % Per ipotesi induttiva, sappiamo che:
    % \begin{equation*}
    %     \New^k = New^k \wedge \Reach^k = Reach^k
    % \end{equation*}

    % Sostituendo in (\ref{th:sound:def_var}), otteniamo
    % \begin{equation}\label{th:sound:def_var_sost}
    %     \begin{cases}
    %         \New^{k+1} = Post(New^k) \setminus Reach^k \\
    %         \Reach^{k+1} = Reach^k \cup New^k
    %     \end{cases}
    % \end{equation}

    % Per dimostrare che $\New^{k+1} = New^{k+1}\ \wedge\ \Reach^{k+1} = Reach^{k+1}$, per la (\ref{th:sound:def_ind}) e la (\ref{th:sound:def_var_sost}), è ora sufficiente dimostrare che
    % \begin{numcases}{}
    %     Reach^k \cup New^k = Reach^k \cup Post(Reach^k) \\
    %     Post(New^k) \setminus Reach^k = (Reach^k \cup New^k) \setminus Reach^k 
    % \end{numcases}

    \paragraph{Terminazione}
    Dimostriamo che l'algoritmo termina sempre ritornando True o False.

    Sappiamo che $\mathbb U$ è finito, perché ogni 
    
    stato è una combinazione di un numero finito e costante di variabili, che possono assumere un numero finito di valori.
    Inoltre, $\exists n$ tale per cui:
    $$
    \begin{cases}
          Reach^k \subset Reach^{k+1} & \forall k < n \\
          Reach^k = Reach^{k+1} & \forall k \geq n
    \end{cases}
    $$
    Il motivo per cui è necessaria l'esistenza di tale $n$ è che la serie $(Reach^k)_k$ è crescente (per definizione di $Reach$), discreta (trattandosi di insiemi) e limitata superiormente (poiché $Reach^k \subseteq \mathbb U\ \forall k$).
    
    Di conseguenza, per la definizione di $New$, avremo che $\exists n$ tale che $New^{k+1} = Reach^{k+1} \setminus Reach^k = \varnothing\ \forall k \geq n$.

    Ciò significa che in al più $n$ iterazioni, l'algoritmo esce dal ciclo principale e quando lo fa ritorna True. 

    \paragraph{Assenza di falsi positivi}
    Dimostriamo di seguito che $Reach^k \subseteq Inv\ \forall k$ se l'algoritmo ritorna true.

    È banale vedere che se l'algoritmo ritorna true, allora il ciclo while è terminato dopo $k$ iterazioni perché $New^k = \varnothing$ e in ogni iterazione precedente non si è entrati nella condizione di riga 6. Ovvero, $New^{k'} \cap NotInv = \varnothing\ \forall k' < k$.
    Inoltre, $New^k = \varnothing \subseteq Inv$.

    Di conseguenza è possibile affermare che se l'algoritmo ritorna true allora $New^{k'} \subseteq Inv\ \forall k' \leq k$.

    Procediamo ora per induzione su $k$ per dimostrare che 
    \begin{equation}\label{th:true:tesi}
        New^{k'} \subseteq Inv\ \forall k' \leq k \implies Reach^k \subseteq Inv
    \end{equation}
    
    \subparagraph*{Caso base:}
    Per $k = 0$ abbiamo $Reach^0 = Init = New^0$ per le definizioni di $Reach$ e $New$. Di conseguenza, $New^0 \subseteq Inv \implies Reach^0 \subseteq Inv$ è trivially true.

    \subparagraph*{Caso induttivo ($k+1$):}
    Vogliamo dimostrare induttivamente che
    \begin{equation}\label{th:true:ind:tesi}
        New^{k'} \subseteq Inv\ \forall k' \leq k+1 \implies Reach^{k+1} \subseteq Inv
    \end{equation}
    Possiamo assumere che $New^{k'} \subseteq Inv\ \forall k' \leq k+1$, altrimenti la parte sinistra di (\ref{th:true:ind:tesi}) è falsa e, di conseguenza, il caso induttivo è trivialmente vero.

    Vogliamo dunque dimostrare $Reach^{k+1} \subseteq Inv$ assumendo
    \begin{numcases}{}
        New^{k'} \subseteq Inv\ \forall k' \leq k+1 & \mbox{per quanto appena detto} \label{th:true:ind:ass_1} \\
        New^{k'} \subseteq Inv\ \forall k' \leq k \implies Reach^k \subseteq Inv & \mbox{per ipotesi induttiva} \label{th:true:ind:hp_ind} \\
        Reach^{k+1} = Reach^{k} \cup New^{k+1} & \mbox{per definizione di $New$} \label{th:true:ind:def_new}
    \end{numcases}

    È banale osservare che da (\ref{th:true:ind:ass_1}) seguono
    \begin{numcases}{}
        New^{k'} \subseteq Inv\ \forall k' \leq k \label{th:true:ind:hp_ind:left} \\
        New^{k+1} \subseteq Inv \label{th:true:ind:new}
    \end{numcases}

    La (\ref{th:true:ind:hp_ind}) e la (\ref{th:true:ind:hp_ind:left}) implicano
    \begin{equation}\label{th:true:ind:hp_ind:right}
        Reach^k \subseteq Inv
    \end{equation}

    Infine, dalla (\ref{th:true:ind:def_new}), la (\ref{th:true:ind:new}) e la (\ref{th:true:ind:hp_ind:right}) segue la tesi induttiva (\ref{th:true:ind:tesi}).
    
    \paragraph{Assenza di falsi negativi}
    Dimostriamo di seguito che se l'algoritmo ritorna false allora $\exists k \mbox{ s.t. } Reach^k \cap NotInv \neq \varnothing$.

    È banale vedere che l'unico caso in cui l'algoritmo ritorna false è quando all'iterazione $k$ si verifica la condizione $New^k \cap NotInv \neq \varnothing$.
    
    \section{Counterexample search}
    \subsection{Idea di base}
    Durante la ricerca degli stati raggiungibili, ad ogni iterazione viene memorizzato l'insieme $New$ in una lista.
    Se a una certa iterazione $New \cap NotInv = \varnothing$, invece di aggiungere in coda alla lista $New$ aggiungiamo $New \cap NotInv$, ovvero gli stati che non rispettano l'invariante.
    A partire da questa lista costruiamo ricorsivamente la trace degli stati che portano il sistema a non rispettare l'invariante nel seguente modo:
    \begin{itemize}
        \item \textbf{Caso base:} Se la lista ha un solo elemento E, ritorno uno stato qualsiasi in E;
        \item \textbf{Caso Induttivo:} Altrimenti, 
        \begin{enumerate}
            \item estrae dalla lista l'ultimo elemento \texttt{sym\_target};
            \item sceglie uno stato qualsiasi in \texttt{sym\_target} che chiamiamo \texttt{target};
            \item modifica il nuovo ultimo elemento della lista restringendolo in modo tale che da ogni suo elemento sia raggiungibile \texttt{target} in una iterazione;
            \item invoca ricorsivamente sulla lista modificata e concatena il risutato con \texttt{target}.
        \end{enumerate}
    \end{itemize}

    Questo algoritmo consente di poter sempre scegliere un elemento qualsiasi nella regione in coda alla lista.
    La modifica all'ultimo elemento prima della ricorsione serve a garantire il mantenimento di questa proprietà.

    \subsection{Implementazione}
    Di seguito viene riportato lo pseudocodice dell'algoritmo per creare la trace degli stati che portano il sistema a non rispettare l'invariante:
    \begin{minted}[bgcolor=bg, breaklines, fontsize=\small, mathescape=true, escapeinside=||, linenos]{python}
function CreateTrace(SymTrace)
    if len(SymTrace) == 0 return []
    if len(SymTrace) == 1 return [PickOneState(SymTrace[0])]

    SymTarget |$\leftarrow$| Symtrace.pop()
    Target |$\leftarrow$| PickOneState(SymTarget)
    SymTrace.last = SymTrace.last |$\cap$| Pre(Target)
    Trace |$\leftarrow$| CreateTrace(SymTrace)
    Trace |$\leftarrow$| Trace + Target
    return Trace
end function
    \end{minted}
    Considerando Trace la lista di stati ottentua dalla funzione precedente, per ottenere gli input che portano il sistema da uno stato presente in Trace al successivo, basta utilizzare la funzione \mintinline{python}{GetInputsBetweenStates(State1, State2)} ed applicarla ad ogni coppia di stati adiacenti nella lista.
    
    \subsection{Dimostrazione di correttezza}
    \begin{itemize}
        \item Precondizione:
            \begin{itemize}
                \item SymTrace è una lista di regioni disgiunte;
                \item ogni elemento in ogni regione in SymTrace è raggiungibile da almeno un elemento della regione precedente se esiste.
            \end{itemize}
        \item Postcondizione: La funzione ritorna una lista Trace di stati tali che:
        \begin{itemize}
            \item SymTrace[i] = R $\implies$ Trace[i] $\in$ R;
            \item Trace[i+1] $\in$ Post(\{Trace[i]\}).
        \end{itemize}
    \end{itemize}

\end{document}
