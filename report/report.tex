\documentclass[12pt]{article}
\usepackage[a4paper, text={6.5in,9in}]{geometry}
% \usepackage[utf8]{inputenc}
% \usepackage{graphicx}
% \graphicspath{ {immagini} }

% \usepackage{titling}

\usepackage{hyperref}
\hypersetup{
    colorlinks,
     citecolor=black,
     filecolor=black,
     linkcolor=black,
     urlcolor=black
 }

% \usepackage{fancyhdr}
% \pagestyle{fancy}

\usepackage{amsmath}
\usepackage{amssymb}
\usepackage{mathtools}
\usepackage{dsfont}
% \newcommand*{\Z}{\mathds{Z}}

\usepackage{minted, xcolor}
%\usemintedstyle{monokai}
\definecolor{bg}{HTML}{F0F0F0}
\usepackage[defaultmono]{droidsansmono}
% \usepackage[T1]{fontenc}

% \pretitle{%
%   \begin{center}
%   \LARGE
%   \includegraphics[width=6cm]{logo-unipd}\\[\bigskipamount]
% }
% \posttitle{\end{center}}

\title{\textbf{Università di Padova \\ Formal Methods for Cyber-Physical Systems project report}}
\author{Alberto Lazari - 1216747\\}
\date{Giugno 2022 - A.A. 2021-2022}

\renewcommand*\contentsname{Indice}

\begin{document}
    \maketitle
    \pagebreak

    \tableofcontents
    \pagebreak

    \section{Notazione}
    \begin{description}
        \item[$Post$]: funzione che rappresenta la regione di stati raggiungibili da una regione data applicando un solo passo di transizione.
    \end{description}

    \section{Come funziona?}
    \subsection{Idea di base}
    L'obiettivo di questo algoritmo è, dati una regione di stati iniziali $Init$, una funzione di transizione $Trans$ e un invariante $Inv$, decidere se $Inv$ è verificato in tutti gli stasti raggiungibili a partire da $Init$ applicando $Trans$.

    L'idea per fare ciò è trovare una rappresentazione simbolica di tutti e soli gli stati raggiungibili dal sistema.
    Così facendo risulta semplice verificare se in alcuni di questi stati l'invariante non è verificato.
    È possibile utilizzare un'ottimizzazione per evitare di calcolare tutti gli stati nel caso in cui nel processo vengano trovati degli stati che non soddisfano l'invariante.

    Per fare ciò, utilizziamo una variabile $Reach$ che rappresenta gli stati raggiungibili dal sistema e la aggiorniamo iterativamente.
    L'idea di base è che all'iterazione $t$, $Reach$ rappresenta tutti gli stati raggiungibili a partire da $Init$ e applicando $Trans$ al più $t$ volte.
    L'algoritmo si interrompe quando si verifica una delle due seguenti condizioni:
    \begin{enumerate}
        \item $Post(Reach) \subseteq Reach $: abbiamo trovato tutti i possibili stati raggiungibili;
        \item $Reach \cap \neg Inv \neq \emptyset$: abbiamo trovato alcuni stati raggiungibili in cui non vale l'invariante.
    \end{enumerate}

    %TODO
    \textbf{TODO Disegnino}

    \subsection{Implementazione}
    Di seguito viene riportato lo pseudocodice dell'algoritmo già descritto:

    %TOEND
    \begin{minted}[bgcolor=bg, breaklines, fontsize=\small, mathescape=true, escapeinside=||]{python}
function IsInvariantRespected(Init, Trans, Inv)
    |$NotInv \leftarrow \neg Inv$|
    |$Reach \leftarrow Init$|
    |$New \leftarrow Init$|
    while |$New \neq \varnothing$| do
        if |$New \cap NotInv \neq \varnothing$| then
            return False
        end if
        |$New \leftarrow Post(New, Trans) \setminus Reach$|
        |$Reach \leftarrow Reach \cup New$|
    end while
    return True
end function
    \end{minted}

    Per l'Implementazione effettiva in python è sufficiente tradurre le seguenti istruzioni:
    \begin{itemize}
        \item $A \leftarrow B$ diventa: \mintinline{python}{A = B}
        \item $\neg A$ diventa: \mintinline{python}{~A}
        \item $A \neq \varnothing$ diventa: \mintinline{python}{not A.is_false()}
        \item $A \cap B \neq \varnothing$ diventa: \mintinline{python}{A.intersected(B)}
        \item $Post(A, Trans)$ diventa: \mintinline{python}{model.post(A)}
        \item $A \setminus B$ diventa: \mintinline{python}{A - B}
        \item $A \cup B$ diventa: \mintinline{python}{A + B}
    \end{itemize}

    %TODO spiegare operatori

    %TODO maybe pensare se mettere il codice easy python
    
    \section{Perché funziona?}
    Definiamo $\mathbb U$ l'insieme di tutti i possibili stati del modello. Sappiamo che l'insieme è finito, perché ogni stato è una combinazione di un numero finito e costante di variabili, che possono assumere un numero finito di valori.
    Definiamo $Inv \subseteq \mathbb U$.
    Definiamo $Reach^k$ in modo tale che:
    \begin{itemize}
        \item $Reach^0 = Init$
        \item $Reach^{k + 1} = Post(Reach^k)$
    \end{itemize}

    Vogliamo dimostrare che l'algoritmo ritorna true sse: $Reach^k \subseteq Inv\ \forall k$.
    Definiamo $NotInv \vcentcolon= \mathbb U \setminus Inv \subseteq \mathbb U$.

    \subsection{Dimostrazione}
\end{document}
