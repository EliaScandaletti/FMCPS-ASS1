\documentclass[12pt]{article}
\usepackage[a4paper, text={6.5in,9in}]{geometry}
% \usepackage[utf8]{inputenc}
% \usepackage{graphicx}
% \graphicspath{ {immagini/} }

% \usepackage{titling}

% \usepackage{hyperref}
% \hypersetup{
%     colorlinks,
%     citecolor=black,
%     filecolor=black,
%     linkcolor=black,
%     urlcolor=black
% }

% \usepackage{fancyhdr}
% \pagestyle{fancy}

% \usepackage{amsmath}
% \usepackage{dsfont}
% \newcommand*{\Z}{\mathds{Z}}

% %\usepackage{minted, xcolor}
% %\usemintedstyle{monokai}
% %\definecolor{bg}{HTML}{282828}
% %\usepackage[defaultmono]{droidsansmono}
% \usepackage[T1]{fontenc}

% \pretitle{%
%   \begin{center}
%   \LARGE
%   \includegraphics[width=6cm]{logo-unipd}\\[\bigskipamount]
% }
% \posttitle{\end{center}}

\title{\textbf{Università di Padova \\ Formal Methods for Cyber-Physical Systems project report}}
\author{Alberto Lazari - 1216747\\}
\date{Giugno 2022 - A.A. 2021-2022}

\renewcommand*\contentsname{Indice}

\begin{document}
	\maketitle
	\pagebreak

	\tableofcontents
	\pagebreak

    \section{Notazione}
    \begin{description}
        \item[$Post$]: funzione che rappresenta la regione di stati raggiungibili da una regione data applicando un solo passo di transizione.
    \end{description}

	\section{Come funziona?}
	\subsection{Idea di base}
						L'obiettivo di questo algoritmo è, dati una regione di stati iniziali $Init$, una funzione di transizione $Trans$ e un invariante $Inv$, decidere se $Inv$ è verificato in tutti gli stasti raggiungibili a partire da $Init$ applicando $Trans$.

						L'idea per fare ciò è trovare una rappresentazione simbolica di tutti e soli gli stati raggiungibili dal sistema.
						Così facendo risulta semplice verificare se in alcuni di questi stati l'invariante non è verificato.
						È possibile utilizzare un'ottimizzazione per evitare di calcolare tutti gli stati nel caso in cui nel processo vengano trovati degli stati che non soddisfano l'invariante.

		Per fare ciò, utilizziamo una variabile $Reach$ che rappresenta gli stati raggiungibili dal sistema e la aggiorniamo iterativamente.
    L'idea di base è che all'iterazione $t$, $Reach$ rappresenta tutti gli stati raggiungibili a partire da $Init$ e applicando $Trans$ al più $t$ volte.
    L'algoritmo si interrompe quando si verifica una delle due seguenti condizioni:
    \begin{enumerate}
        \item $Post(Reach) \subseteq Reach $: abbiamo trovato tutti i possibili stati raggiungibili;
        \item $Reach \cap \neg Inv \neq \emptyset$: abbiamo trovato alcuni stati raggiungibili in cui non vale l'invariante.
    \end{enumerate}

	%TODO
	\textbf{TODO Disegnino}

	\subsection{Implementazione}
    
	\section{Perché funziona?}
	Spiegazione formale
\end{document}
